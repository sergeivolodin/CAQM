\documentclass[a4paper]{article}
\usepackage[a4paper, left=25mm, right=25mm, top=25mm, bottom=25mm]{geometry}
%\geometry{paperwidth=210mm, paperheight=2000pt, left=5pt, top=5pt}
\usepackage[utf8]{inputenc}
\usepackage[english,russian]{babel}
\usepackage{indentfirst}
\usepackage{tikz} %Рисование автоматов
\usetikzlibrary{automata,positioning,arrows,trees}
\usepackage{amsmath}
\usepackage{enumerate}
\usepackage[makeroom]{cancel} % зачеркивание
\usepackage{multicol,multirow} %Несколько колонок
\usepackage{hyperref}
\usepackage{tabularx}
\usepackage{amsfonts}
\usepackage{amssymb}
\DeclareMathOperator*{\argmin}{arg\,min}
\usepackage{wasysym}
\date{}
\title{Nonconvexity certificate in $b_i=0$ case}

\begin{document}
\maketitle
\subsection*{Cases}

Let $f\colon \mathbb{R}^n\to\mathbb{R}^m$ be a quadratic map: $f_i(x)=x^TA_ix$, $A_i=A_i^T$.

Consider $c\in\mathbb{R}^m$ and boundary points $\partial F_c$:
$$c\cdot f(x)\to \min\limits_c$$

Where $A=\sum c_iA_i$. Assuming $A\geqslant 0$.

Minimization leads to $Ax=0$. The following cases hold:
\begin{enumerate}
\item $RgA=n$. $x=0$ is a unique solution
\item $RgA=n-1$. $x=\alpha e$, $f(x)=\alpha^2 f(e)$
\item $RgA=n-2$. In this case $x=\alpha_1 e^1+\alpha_2 e^2$. Consider $f(x)=\alpha_1^2 f_{11}+2\alpha_1\alpha_2f_{12}+\alpha_2^2f_{22}$.
\begin{enumerate}
\item $f_{11}$, $f_{22}$, $f_{12}$ are linearly independent. In this case $\partial F_c$ is nonconvex
\item $Rg ||f_{11} f_{12} f_{22}||=1$. $\partial F_c$ convex.
\end{enumerate}
\end{enumerate}


\subsection*{Result}
If exist $c\in\mathbb{R}^m$:
\begin{enumerate}
\item $RgA=n-2$
\item $A\geqslant 0$
\item $f_{ij}$ are linearly independent
\end{enumerate}
Then $F=\mbox{Im}f$ is nonconvex.
\end{document}