\documentclass[a4paper]{article}
\usepackage[a4paper, left=5mm, right=5mm, top=5mm, bottom=5mm]{geometry}
%\geometry{paperwidth=210mm, paperheight=2000pt, left=5pt, top=5pt}
\usepackage[utf8]{inputenc}
\usepackage[english,russian]{babel}
\usepackage{indentfirst}
\usepackage{tikz}
\usepackage{cancel}
\usetikzlibrary{automata,positioning,arrows}
\usepackage{amsmath}
\usepackage{enumerate}
\usepackage{hyperref}
\usepackage{amsfonts}
\usepackage{amssymb}
\DeclareMathOperator*{\argmin}{arg\,min}
\DeclareMathOperator*{\argmax}{arg\,max}
\usepackage{wasysym}
\title{On the feasibility for the system of quadratic equations\\MATLAB Library}
\date{}
\author{Anatoly Dymarsky, Elena Gryazina, Boris Polyak, Sergei Volodin}
\newcommand{\matrixl}{\left|\left|}
\newcommand{\matrixr}{\right|\right|}

\newcommand{\peq}{\mathrel{+}=}
\newcommand{\meq}{\mathrel{-}=}
\newcommand{\deq}{\mathrel{:}=}
\newcommand{\VC}{\mbox{VC}}
\newcommand{\plpl}{\mathrel{+}+}
\newcommand{\sign}{\mbox{sign}\,}
\newcommand{\F}{\mathcal{F}}
\newcommand{\R}{\mathbb{R}}
\newcommand{\conv}{\mbox{conv}\,}
\newcommand{\E}{\mathbb{E}}
\newcommand{\D}{\mathbb{D}}

\usepackage{amsthm}
\theoremstyle{definition}
\newtheorem{definition}{Definition}[section]

% пустое слово
\def\eps{\varepsilon}

% регулярные языки
\def\eqdef{\overset{\mbox{\tiny def}}{=}}
\newcommand{\niton}{\not\owns}

\begin{document}
\maketitle
\section{Notations}
The goal of the project is to solve a number of tasks for quadratic maps, which are
\begin{enumerate}
\item (Real case) The map $f\colon \mathbb{R}^n\to\mathbb{R}^m$ s.t. $$f_i(x)=x^TA_ix+2b_i^Tx,\, A_i=A_i^T$$
\item (Complex case) The map $f\colon \mathbb{C}^n\to\mathbb{R}^m$ s.t. $$f_i(x)=x^*A_ix+b_i^*x+x^*b_i,\, A_i=A_i^*$$
Where $\cdot^*$ is Hermitian transpose.
\end{enumerate}
We use the following notations:
\theoremstyle{definition}
\begin{definition} For a vector $c\in\mathbb{R}^n$ and tuple of matrices $(A_1,...,A_n)$ (or vectors) the dot product is defined as following: $$c\cdot A=\sum\limits_{i=1}^nc_iA_i$$
\end{definition}
\begin{definition} The image of $f$ is denoted as $F$:
	$$F=f(\mathbb{R}^n)$$
\end{definition}
\begin{definition} The convex hull of $F$ is denoted as $G$:
	$$G=\conv F$$
\end{definition}
\section{Functions}
\section{Example}
\end{document}