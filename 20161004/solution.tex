\documentclass[a4paper]{article}
\usepackage[a4paper, left=25mm, right=25mm, top=25mm, bottom=25mm]{geometry}
%\geometry{paperwidth=210mm, paperheight=2000pt, left=5pt, top=5pt}
\usepackage[utf8]{inputenc}
\usepackage[english,russian]{babel}
\usepackage{indentfirst}
\usepackage{tikz} %Рисование автоматов
\usetikzlibrary{automata,positioning,arrows,trees}
\usepackage{amsmath}
\usepackage{enumerate}
\usepackage[makeroom]{cancel} % зачеркивание
\usepackage{multicol,multirow} %Несколько колонок
\usepackage{hyperref}
\usepackage{tabularx}
\usepackage{amsfonts}
\usepackage{amssymb}
\DeclareMathOperator*{\argmin}{arg\,min}
\usepackage{wasysym}
\date{}
\title{Draft.pdf}

\begin{document}
\maketitle
\subsection*{Equation 0.19}
Consider $A=S\Lambda S^T$, $S^TS=E$, $x_0$ is a simple zero eigenvector of $A$: $Ax_0=0$, $||x_0||=1$.

$A^{-1}=S\Lambda_1S^T$, $\Lambda=(\lambda_1,...,\lambda_{n-1},0)$, $\Lambda_1=(\lambda_1^{-1},...,\lambda_{n-1}^{-1},0)$.

Equation (0.19):
$$\dot{x_0}=-A^{-1}\dot{A}x_0$$

Consider $A^{-1}\dot{A}x_0=S\Lambda_1S^T(\dot{S}\Lambda S^T+S\dot{\Lambda} S^T+S\Lambda \dot{S}^T)x_0\boxed{=}$.

Consider $x_0=Sy_0$, where $y_0=(0,0,...,1)$. Therefore, $0=\dot{y}_0=\dot{S}^Tx_0+S^T\dot{x}_0$

Going back to 0.19, the part $\Lambda S^Tx_0=\Lambda y_0=0$, another part $\dot{\Lambda}S^Tx_0=0$. Consequently,

$\boxed{=}S\Lambda_1S^TS\Lambda \dot{S}^Tx_0=S\Lambda_1\Lambda\dot{S}^Tx_0=-S\Lambda_1\Lambda S^T\dot{x}_0=-\sum\limits_{i=1}^{n-1}s_is_i^T\dot{x}_0=-(E-x_0x_0^T)\dot{x}_0=-\dot{x}_0+x_0x_0^T\dot{x}_0\boxed{=}$.

Taking a derivative $||x_0||=1$, we get $x_0^T\dot{x_0}=0$, therefore,

$\boxed{=}-\dot{x}_0$
\end{document}