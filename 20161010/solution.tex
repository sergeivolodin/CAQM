\documentclass[a4paper]{article}
\usepackage[a4paper, left=25mm, right=25mm, top=25mm, bottom=25mm]{geometry}
%\geometry{paperwidth=210mm, paperheight=2000pt, left=5pt, top=5pt}
\usepackage[utf8]{inputenc}
\usepackage[english,russian]{babel}
\usepackage{indentfirst}
\usepackage{tikz} %Рисование автоматов
\usetikzlibrary{automata,positioning,arrows,trees}
\usepackage{amsmath}
\usepackage{enumerate}
\usepackage[makeroom]{cancel} % зачеркивание
\usepackage{multicol,multirow} %Несколько колонок
\usepackage{hyperref}
\usepackage{tabularx}
\usepackage{amsfonts}
\usepackage{amssymb}
\usepackage{amsmath}
\DeclareMathOperator{\Tr}{Tr}
\DeclareMathOperator{\conv}{conv}
\DeclareMathOperator*{\argmin}{arg\,min}
\usepackage{wasysym}
\date{}
\title{On the feasibility for the system of quadratic equations, explanations}

\begin{document}
\maketitle
\subsection*{Theorem 3.2 (Sufficient condition)}
Consider $f\colon \mathbb{R}^n\to\mathbb{R}^m$, s.t. $f_i(x)=x^TA_ix+2b_i^Tx$, $A_i=A_i^T$. Define $F=f(\mathbb{R}^n)$.

Then why $A=\inf \limits_{y\in F} (c,y)=\inf\limits_{y\in\conv F} (c,y)=B$?

\begin{enumerate}
\item First, $F\subseteq \conv F$, therefore, $B\leqslant A$.
\item Secondly, let $y_k\in \conv F$ be a sequence s.t. $g_k=(c,y_k)\underset{k\to\infty}{\longrightarrow} B$. $y_k=\sum\limits_{i=1}^{n_k}\alpha^k_iy^k_i$.

$g_k(c,y_k)=\sum\limits_{i=1}^{n_k}\alpha^k_i(c,y^k_i)=\sum\limits_{i=1}^{n_k}\alpha^k_i g^k_i$. Define $g^k_0=\min\limits_{i\in \overline{1,n_k}}g^k_i$. Then $B\leqslant g^k_0\leqslant g^k$. Therefore, $g^k_0\to B$ also. This way, we have constructed a sequence $y^k_0\in F$ s.t. $(c,y^k_0)\to B$, therefore, $A\leqslant B$.
\end{enumerate}

\subsection*{Finding $c$ provided $d$}
Let $H\colon \mathbb{R}^{n+1, n+1}\to \mathbb{R}^n$ be a map s.t. $H_i(X)=\Tr(H_iX)$, $$H_i=\left|\left|
\begin{array}{cc}
A_i & b_i\\
b_i^T & 0
\end{array}
\right|\right|^{\Box}$$.

Consider a boundary point $X$, which is a solution of:
$$\sup\limits_{\begin{cases}
H(X)=y^0+td\\
X\geqslant 0\\
X_{n+1,n+1}=1
	\end{cases}} t$$.

Define $f(t,X)=t$, $D_0=\{(t,X)\big| X\geqslant 0,\, X_{n+1,n+1}=1 \}$, $D_1=\{(t,X)\big| H(X)=y^0+td\}$.

Then supremum is equivalent to

$$\sup\limits_{(t,X)\in D_0\cap D_1}f(t,X)$$

Define a Lagrange function $L(c,t,X)=\underbrace{t}_{f(t,X)}+\sum\limits_{i=1}^m c_i(y^0_i+td_i-H_i(X))$

Then the dual function is $g(c)=\sup\limits_{(t,X)\in D_0} L(c,t,X)$.

Because $L=t(1+\sum\limits_{i=1}^m c_id_i)+\sum\limits_{i=1}^n c_i(y^0_i-H_i(X))$, $g=+\infty$ when $(c,d)\neq -1$. From this point we assume that $\boxed{(c,d)=-1}$.

Now, $g(c)=\sup\limits_{X_{n+1,n+1}=1,X\geqslant 0} (c,y^0-H(X))=(c,y^0)+\sup\limits_{y\in\conv F} -(c,y)=(c,y^0)-\inf\limits_{y\in\conv F} (c,y)$.

Then the dual problem is
$$g(c)\to\inf\limits_{(c,d)=-1}$$
\end{document}